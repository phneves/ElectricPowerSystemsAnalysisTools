\chapter{Discussões e an\'alise de resultados}
\section{Composição do trabalho}
Neste trabalho, foi abordado uma introdução à teoria de Fluxo de Potência e um dos métodos de resolução bastante importante, o método de Newton, bem como algumas otimizações que se baseiam no desacoplamento das variáveis do problema, como descrito em \ref{SubsectionMetodoNewtonDesacopladoVariaveis}. As três otimizações, descritas na seção \ref{SubsectionMetodoDeNewtonDesacoplado}, se baseiam em simplificações na matriz Jacobiana, como pode ser visto em \ref{HNML_eq_rapido} para o método desacoplado, \ref{H_resolvido_Nmodificado} e \ref{L_resolvidoNmodificado} para o método desacoplado versão modificada e \ref{Blinha} e \ref{Bduaslinhas} para o método desacoplado rápido. No capítulo \ref{SectionEstudosDeCaso} há uma comparação entre dois métodos, desacoplado e desacoplado rápido para uma rede pequena, demostrada em \ref{SectionRedePequena}. Outra rede verificada, foi a de 14 barras e 20 ramos em \ref{SectionRede14barras}. Neste estudo não houve equacionamento por ser, praticamente, inviável calcular manualmente. A simulação convergiu após 4 iterações e levou $0,3942s$ para o método de Newton e para o método de Newton desacoplado, ela convergiu em $0,3365s$, embora com um número de iterações muito maior.
\section{Performance}
Comparando-se os Métodos de Newton, é notório a simplificação trazida pelo pelos métodos desacoplados, especialmente o desacoplado rápido. Embora não fique tão claro em redes menores, fica bastante evidente em redes maiores, como em \ref{SectionRede14barras}. O número de iterações é maior, mas com esforço computacional menor, o tempo de convergência é menor (como o sistema operacional não se mantem dedicado exclusivamente para o matlab, a comparação de tempo pode variar).


