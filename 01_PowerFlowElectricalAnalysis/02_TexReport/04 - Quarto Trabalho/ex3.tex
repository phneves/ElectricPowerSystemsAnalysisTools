\chapter{C\'odigo comentado}
\label{SectionCodigo}

Os códigos fonte desse trabalho bem como histórico de modificação, podem ser encontrados no endereço: \textit{\href{https://github.com/phneves/ElectricPowerSystemsAnalysisTools}{https://github.com/phneves/ElectricPowerSystemsAnalysisTools}}.\\
Entrada de dados do estimador de estados.
\begin{minted}[mathescape, style = autumn,
    frame = single, fontsize=\footnotesize]{matlab}
num = 14; 
ybus = ybusppg(num); 
zdata = zdatas(num); %pneves: Get Measurement data
bpq = bbusppg(num); % Get B data
nbus = max(max(zdata(:,4)),max(zdata(:,5))); % Get number of buses
type = zdata(:,2);
z = zdata(:,3); % Measuement values
fbus = zdata(:,4); % From bus
tbus = zdata(:,5); % To bus
Ri = diag(zdata(:,6)); % Measurement Error
V = ones(nbus,1); % Initialize the bus voltages
del = zeros(nbus,1); % Initialize the bus angles
E = [del(2:end); V];   % State Vector
G = real(ybus);
B = imag(ybus);
\end{minted}

\begin{minted}[mathescape, style = autumn,
    frame = single, fontsize=\footnotesize]{matlab}
vi = find(type == 1); % Index of voltage magnitude measurements
ppi = find(type == 2); % Index of real power injection measurements
qi = find(type == 3); % Index of reactive power injection measurements
pf = find(type == 4); % Index of real powerflow measurements
qf = find(type == 5); % Index of reactive powerflow measurements
\end{minted}
Calcula o número de medidas disponíveis.
\begin{minted}[mathescape, style = autumn,
    frame = single, fontsize=\footnotesize]{matlab}
%pneves
nvi = length(vi); % Number of Voltage measurements
npi = length(ppi); % Number of Real Power Injection measurements
nqi = length(qi); % Number of Reactive Power Injection measurements
npf = length(pf); % Number of Real Power Flow measurements
nqf = length(qf); % Number of Reactive Power Flow measurements
iter = 1;
tol = 5;
\end{minted}
%%%%%%%%%%%%%%%%%%%%%%%%%%%%%%%%%%%%%%%%%%%%%%%%%%%%%%%%%%%%%%%%%%%%%%%%%%%%%%%%%%%%%%%%%%%%%%%%
Itera-se comparando com a tolerância desejada. 
\begin{minted}[mathescape, style = autumn,
    frame = single, fontsize=\footnotesize]{matlab}
while(tol > 1e-4)
    %Measurement Function, h 
    h1 = V(fbus(vi),1);
    h2 = zeros(npi,1);
    h3 = zeros(nqi,1);
    h4 = zeros(npf,1);
    h5 = zeros(nqf,1);
\end{minted}
%%%%%%%%%%%%%%%%%%%%%%%%%%%%%%%%%%%%%%%%%%%%%%%%%%%%%%%%%%%%%%%%%%%%%%%%%%%%%%%%%%%%%%%%%%%%%%%%

\begin{minted}[mathescape, style = autumn,
    frame = single, fontsize=\footnotesize]{matlab}

    for i = 1:npi
        m = fbus(ppi(i));
        for k = 1:nbus
            h2(i) = h2(i) + V(m)*V(k)*(G(m,k)*cos(del(m)-del(k)) + ...
                B(m,k)*sin(del(m)-del(k)));
        end
    end
    
    for i = 1:nqi
        m = fbus(qi(i));
        for k = 1:nbus
            h3(i) = h3(i) + V(m)*V(k)*(G(m,k)*sin(del(m)-del(k)) - ...
                B(m,k)*cos(del(m)-del(k)));
        end
    end
    
    for i = 1:npf
        m = fbus(pf(i));
        n = tbus(pf(i));
        h4(i) = -V(m)^2*G(m,n) - V(m)*V(n)*(-G(m,n)*cos(del(m)-del(n))
            - B(m,n)*sin(del(m)-del(n)));
    end
    
    for i = 1:nqf
        m = fbus(qf(i));
        n = tbus(qf(i));
        h5(i) = -V(m)^2*(-B(m,n)+bpq(m,n)) 
            - V(m)*V(n)*(-G(m,n)*sin(del(m)-del(n))
            + B(m,n)*cos(del(m)-del(n)));
    end
    
    h = [h1; h2; h3; h4; h5];
    
    %pneves: Residue
    r = z - h;
\end{minted}
%%%%%%%%%%%%%%%%%%%%%%%%%%%%%%%%%%%%%%%%%%%%%%%%%%%%%%%%%%%%%%%%%%%%%%%%%%%%%%%%%%%%%%%%%%%%%%%%
Monta-se a matriz Jacobiana seguindo \ref{H}.\\
$H_{11}$ Derivada de $V$ com respeito a $\theta$.
\begin{minted}[mathescape, style = autumn,
    frame = single, fontsize=\footnotesize]{matlab}

    H11 = zeros(nvi,nbus-1);

\end{minted}
%%%%%%%%%%%%%%%%%%%%%%%%%%%%%%%%%%%%%%%%%%%%%%%%%%%%%%%%%%%%%%%%%%%%%%%%%%%%%%%%%%%%%%%%%%%%%%%%
$H_{12}$ Derivada de $V$ com respeito a $V$.
\begin{minted}[mathescape, style = autumn,
    frame = single, fontsize=\footnotesize]{matlab}
    H12 = zeros(nvi,nbus);
    for k = 1:nvi
        for n = 1:nbus
            if n == k
                H12(k,n) = 1;
            end
        end
    end
\end{minted}
%%%%%%%%%%%%%%%%%%%%%%%%%%%%%%%%%%%%%%%%%%%%%%%%%%%%%%%%%%%%%%%%%%%%%%%%%%%%%%%%%%%%%%%%%%%%%%%%
$H_{21}$ Derivada de $P$ com respeito a $\theta$.
\begin{minted}[mathescape, style = autumn,
    frame = single, fontsize=\footnotesize]{matlab}
    H21 = zeros(npi,nbus-1);
    for i = 1:npi
        m = fbus(ppi(i));
        for k = 1:(nbus-1)
            if k+1 == m
                for n = 1:nbus
                    H21(i,k) = H21(i,k) + ...
                        V(m)* V(n)*(-G(m,n)*sin(del(m)-del(n)) + ...
                        B(m,n)*cos(del(m)-del(n)));
                end
                H21(i,k) = H21(i,k) - V(m)^2*B(m,m);
            else
                H21(i,k) = V(m)* V(k+1)*(G(m,k+1)*sin(del(m)-del(k+1)) 
                - B(m,k+1)*cos(del(m)-del(k+1)));
            end
        end
    end
\end{minted}
%%%%%%%%%%%%%%%%%%%%%%%%%%%%%%%%%%%%%%%%%%%%%%%%%%%%%%%%%%%%%%%%%%%%%%%%%%%%%%%%%%%%%%%%%%%%%%%%
$H_{22}$ Derivada de $P$ com respeito a $V$.
\begin{minted}[mathescape, style = autumn,
    frame = single, fontsize=\footnotesize]{matlab}    
    H22 = zeros(npi,nbus);
    for i = 1:npi
        m = fbus(ppi(i));
        for k = 1:(nbus)
            if k == m
                for n = 1:nbus
                    H22(i,k) = H22(i,k) + ...
                    V(n)*(G(m,n)*cos(del(m)-del(n)) + ...
                    B(m,n)*sin(del(m)-del(n)));
                end
                H22(i,k) = H22(i,k) + V(m)*G(m,m);
            else
                H22(i,k) = V(m)*(G(m,k)*cos(del(m)-del(k)) + ...
                B(m,k)*sin(del(m)-del(k)));
            end
        end
    end
\end{minted}
%%%%%%%%%%%%%%%%%%%%%%%%%%%%%%%%%%%%%%%%%%%%%%%%%%%%%%%%%%%%%%%%%%%%%%%%%%%%%%%%%%%%%%%%%%%%%%%%
$H_{31}$ Derivada de $Q$ com respeito a $\theta$.
\begin{minted}[mathescape, style = autumn,
    frame = single, fontsize=\footnotesize]{matlab}    
    H31 = zeros(nqi,nbus-1);
    for i = 1:nqi
        m = fbus(qi(i));
        for k = 1:(nbus-1)
            if k+1 == m
                for n = 1:nbus
                    H31(i,k) = H31(i,k) + ...
                    V(m)* V(n)*(G(m,n)*cos(del(m)-del(n)) + ...
                    B(m,n)*sin(del(m)-del(n)));
                end
                H31(i,k) = H31(i,k) - V(m)^2*G(m,m);
            else
               H31(i,k) = V(m)* V(k+1)*(-G(m,k+1)*cos(del(m)-del(k+1)) 
               - B(m,k+1)*sin(del(m)-del(k+1)));
            end
        end
    end
\end{minted}
%%%%%%%%%%%%%%%%%%%%%%%%%%%%%%%%%%%%%%%%%%%%%%%%%%%%%%%%%%%%%%%%%%%%%%%%%%%%%%%%%%%%%%%%%%%%%%%%
$H_{32}$ Derivada de $Q$ com respeito a $V$.
\begin{minted}[mathescape, style = autumn,
    frame = single, fontsize=\footnotesize]{matlab}    
    H32 = zeros(nqi,nbus);
    for i = 1:nqi
        m = fbus(qi(i));
        for k = 1:(nbus)
            if k == m
                for n = 1:nbus
                    H32(i,k) = H32(i,k) + ...
                    V(n)*(G(m,n)*sin(del(m)-del(n)) - ...
                    B(m,n)*cos(del(m)-del(n)));
                end
                H32(i,k) = H32(i,k) - V(m)*B(m,m);
            else
               H32(i,k) = V(m)*(G(m,k)*sin(del(m)-del(k)) 
               - B(m,k)*cos(del(m)-del(k)));
            end
        end
    end
\end{minted}
%%%%%%%%%%%%%%%%%%%%%%%%%%%%%%%%%%%%%%%%%%%%%%%%%%%%%%%%%%%%%%%%%%%%%%%%%%%%%%%%%%%%%%%%%%%%%%%%
$H_{41}$ Derivada de $P$ com respeito a $\theta$.
\begin{minted}[mathescape, style = autumn,
    frame = single, fontsize=\footnotesize]{matlab}    
    H41 = zeros(npf,nbus-1);
    for i = 1:npf
        m = fbus(pf(i));
        n = tbus(pf(i));
        for k = 1:(nbus-1)
            if k+1 == m
                H41(i,k) = V(m)* V(n)*(-G(m,n)*sin(del(m)-del(n)) 
                + B(m,n)*cos(del(m)-del(n)));
            else if k+1 == n
                H41(i,k) = -V(m)* V(n)*(-G(m,n)*sin(del(m)-del(n)) 
                + B(m,n)*cos(del(m)-del(n)));
                else
                    H41(i,k) = 0;
                end
            end
        end
    end
\end{minted}
%%%%%%%%%%%%%%%%%%%%%%%%%%%%%%%%%%%%%%%%%%%%%%%%%%%%%%%%%%%%%%%%%%%%%%%%%%%%%%%%%%%%%%%%%%%%%%%%
$H_{42}$ Derivada de $P$ com respeito a $V$.
\begin{minted}[mathescape, style = autumn,
    frame = single, fontsize=\footnotesize]{matlab}    
    H42 = zeros(npf,nbus);
    for i = 1:npf
        m = fbus(pf(i));
        n = tbus(pf(i));
        for k = 1:nbus
            if k == m
                H42(i,k) = -V(n)*(-G(m,n)*cos(del(m)-del(n)) 
                - B(m,n)*sin(del(m)-del(n))) - 2*G(m,n)*V(m);
            else if k == n
                H42(i,k) = -V(m)*(-G(m,n)*cos(del(m)-del(n)) 
                - B(m,n)*sin(del(m)-del(n)));
                else
                    H42(i,k) = 0;
                end
            end
        end
    end
\end{minted}
%%%%%%%%%%%%%%%%%%%%%%%%%%%%%%%%%%%%%%%%%%%%%%%%%%%%%%%%%%%%%%%%%%%%%%%%%%%%%%%%%%%%%%%%%%%%%%%%
$H_{51}$ Derivada de $Q$ com respeito a $\theta$.
\begin{minted}[mathescape, style = autumn,
    frame = single, fontsize=\footnotesize]{matlab}    
    H51 = zeros(nqf,nbus-1);
    for i = 1:nqf
        m = fbus(qf(i));
        n = tbus(qf(i));
        for k = 1:(nbus-1)
            if k+1 == m
                H51(i,k) = -V(m)* V(n)*(-G(m,n)*cos(del(m)-del(n)) 
                - B(m,n)*sin(del(m)-del(n)));
            else if k+1 == n
                H51(i,k) = V(m)* V(n)*(-G(m,n)*cos(del(m)-del(n)) 
                - B(m,n)*sin(del(m)-del(n)));
                else
                    H51(i,k) = 0;
                end
            end
        end
    end
\end{minted}
%%%%%%%%%%%%%%%%%%%%%%%%%%%%%%%%%%%%%%%%%%%%%%%%%%%%%%%%%%%%%%%%%%%%%%%%%%%%%%%%%%%%%%%%%%%%%%%%
$H_{52}$ Derivada de $Q$ com respeito a $V$.
\begin{minted}[mathescape, style = autumn,
    frame = single, fontsize=\footnotesize]{matlab}    
    H52 = zeros(nqf,nbus);
    for i = 1:nqf
        m = fbus(qf(i));
        n = tbus(qf(i));
        for k = 1:nbus
            if k == m
                H52(i,k) = -V(n)*(-G(m,n)*sin(del(m)-del(n)) 
                + B(m,n)*cos(del(m)-del(n))) 
                - 2*V(m)*(-B(m,n)+ bpq(m,n));
            else if k == n
                H52(i,k) = -V(m)*(-G(m,n)*sin(del(m)-del(n)) 
                + B(m,n)*cos(del(m)-del(n)));
                else
                    H52(i,k) = 0;
                end
            end
        end
    end

\end{minted}
%%%%%%%%%%%%%%%%%%%%%%%%%%%%%%%%%%%%%%%%%%%%%%%%%%%%%%%%%%%%%%%%%%%%%%%%%%%%%%%%%%%%%%%%%%%%%%%%
Matriz Jacobiana de medições é montada
\begin{minted}[mathescape, style = autumn,
    frame = single, fontsize=\footnotesize]{matlab}
    H = [H11 H12; H21 H22; H31 H32; H41 H42; H51 H52];
\end{minted}
\section{Equação normal}
\label{SectionEqNormal}
%%%%%%%%%%%%%%%%%%%%%%%%%%%%%%%%%%%%%%%%%%%%%%%%%%%%%%%%%%%%%%%%%%%%%%%%%%%%%%%%%%%%%%%%%%%%%%%%
\begin{minted}[mathescape, style = autumn,
    frame = single, fontsize=\footnotesize]{matlab}
    
    % pneves: Gain Matrix, Gm
    Gm = H'*inv(Ri)*H;
    
    %pneves:  Objective Function
    J = sum(inv(Ri)*r.^2);  
    
    %pneves:  State Vector
    dE = inv(Gm)*(H'*inv(Ri)*r);
    E = E + dE;
    del(2:end) = E(1:nbus-1);
    V = E(nbus:end);
    iter = iter + 1;
    tol = max(abs(dE));
end

\end{minted}
\section{Tableau esparso}
\label{SectionTableau}
\begin{minted}[mathescape, style = autumn,
    frame = single, fontsize=\footnotesize]{matlab}
%pneves: Tableau sparse 
    if iter == 1
        [linhasH , colunasH] = size(H);
        HZeros = zeros(colunasH,colunasH);
    end
    MatrizH = [Ri H; H' HZeros];
    if iter == 1
        [linhasLambda , colunasLambda] = size(MatrizH);
        VetorLambda = zeros(linhasH,1);
        dE = zeros(colunasH,1);
    end
    MatrizLambda = [VetorLambda ; dE];
    Rzeros = zeros(colunasH,1);
    MatrizR = [r ; Rzeros];
    MatrizLambda = inv(MatrizH)*MatrizR;
    dE = MatrizLambda((linhasH+1):end);
    %dE = MatrizLambda(42:end);
    E = E + dE;
    del(2:end) = E(1:nbus-1);
    V = E(nbus:end);
    iter = iter + 1;
    tol = max(abs(dE));
end
\end{minted}

\section{Resultados}
%%%%%%%%%%%%%%%%%%%%%%%%%%%%%%%%%%%%%%%%%%%%%%%%%%%%%%%%%%%%%%%%%%%%%%%%%%%%%%%%%%%%%%%%%%%%%%%%
\begin{minted}[mathescape, style = autumn,
    frame = single, fontsize=\footnotesize]{matlab}
CvE = diag(inv(H'*inv(Ri)*H)); % Covariance matrix
Del = 180/pi*del;
E2 = [V Del]; % Bus Voltages and angles
disp('-------- State Estimation ------------------');
disp('--------------------------');
disp('| Bus |    V   |  Angle  | ');
disp('| No  |   pu   |  Degree | ');
disp('--------------------------');
for m = 1:nbus
    fprintf('%4g', m); 
    fprintf('  %8.4f', V(m)); 
    fprintf('   %8.4f', Del(m)); 
    fprintf('\n');
end
disp('---------------------------------------------');

\end{minted}


