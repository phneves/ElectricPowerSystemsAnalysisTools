\chapter{Discussões e an\'alise de resultados}
\section{Composição do trabalho}
Neste trabalho, foi abordado uma introdução à teoria de Estimadores de Estado com métodos da Equação Normal e Tableau Esparso.\\
A rede 14 barras IEEE foi escolhida para testar convergencia dos dois métodos, apresentados nas seções \ref{SectionEqNormal} e \ref{SectionTableau}. A partir de medidas da rede, como na seção \ref{SectionDados}, foi possivel estimar o estado atual da rede 14 barras IEEE, como mostrado em \ref{SectionResultados}.\\


\section{Performance}
Os dois métodos, explicados nas seções \ref{SectionEqNormal} e \ref{SectionTableau}, convergiram para a mesma solução, que pode ser verificado em \ref{SectionResultados}. O método Tableau Esparso, com leve vantagem computacional, por evitar o calculo da matriz $G$, como em \ref{G}, aliviando o método de calculo.\\
A economia se deve, também, ao fato da matriz gerada ser bastante esparsa.\cite{Mohamad}.
